\documentclass[a4paper, twocolumn]{article}
\topmargin = -2cm
\oddsidemargin = -1cm
\textwidth=16.8cm
\textheight=25.2cm
\usepackage[utf8]{inputenc}
\usepackage{graphicx}
\usepackage{enumitem}
\setlist{nolistsep}

\begin{document}

\newcounter{chapter}
\setcounter{chapter}{3}
\setcounter{page}{220}

of marking is a convenient and concise representation of
the specific and basic data about the product.

\begin{figure}[h]
\centering
\includegraphics[width=0.8\linewidth]{Product with alphanumeric marking.jpg}
\caption{Product with alphanumeric marking}
\label{fig:pam}
\end{figure}

\begin{figure}[h]
\centering
\includegraphics[width=0.8\linewidth]{Product with Data Matrix code.jpg}
\caption{Product with Data Matrix code}
\label{fig:pdm}
\end{figure}

Based on the general formulation of the problem, the
following subtasks can be distinguished, which must be
solved by the system:

\begin{enumerate}
    \newcounter{i}
    \addtocounter{i}{1}
    \item[\arabic{i})] \mbox{detection and recognition of the type of marking;}
    \addtocounter{i}{1}
    \item[\arabic{i})] marking recognition;
    \addtocounter{i}{1}
    \item[\arabic{i})] identifying possible marking problems.
\end{enumerate}

In general, the following requirements can be imposed
on the system that will solve the assigned tasks:

\begin{itemize}
    \item  \textbf{High speed of work.} The production line is moving
    very quickly, so the detection of defects should be
    carried out with minimal delays;
    \item  \textbf{Autonomy.} The system should minimize operator
    involvement in the quality control process;
    \item \textbf{Universality.} The system should be configured to
    recognize the marking of any product;
    \item \textbf{Adaptability.} The system must work under any
conditions that occur in production (for example,
insufficient lighting, personnel errors, etc.).
\end{itemize}

Let’s list the main problems with marking [2]:

\begin{enumerate}
    \newcounter{j}
    \setcounter{j}{0}
    \addtocounter{j}{1}
    \item[\arabic{j})]  \textbf{lack of ink:} in case of receipt of products on the
production line without marking, the system must
conclude that there is no ink;
    \addtocounter{j}{1}
    \item[\arabic{j})]  \textbf{camera shift:} if no data on the recognition results
are received from the neural network modules,
but the system knows that movement along the
production line has begun, then it must conclude
that the camera has shifted;
    \addtocounter{j}{1}
    \item[\arabic{j})] \textbf{bad marking:} marking was found and recognized,
but did not match the template representation.
In this case, the system must conclude that the
marking is incorrect;
    \addtocounter{j}{1}
    \item[\arabic{j})] \textbf{unreadable marking:} if the marking is blurry and
cannot be recognized, it is necessary to stop the
production line and report the error to the operator.
\end{enumerate}  
  
For problems 1,3 and 4, it is necessary to screen out
products that have a problem marking. The occurrence of
these problems implies a complete stop of the production
line movement and reporting to the operator about the
problem.

\begin{center}
    \Roman{chapter}. OVERVIEW OF EXISTING APPROACHES
\end{center}
\addtocounter{chapter}{1}

Despite the existing interest in the autonomy of production processes and the indisputable advantages that its
implementation entails, tasks similar to those described
are solved in a large number of cases with the participation of a person. The operator simply checks a part of
the production periodically and randomly. This approach
has disadvantages:

\begin{itemize}
    \item only a small part of the production passes inspection, so there is a possibility that the defective
marking will be missed;
    \item the speed of a person’s reaction to an emergency
situation may be insufficient;
    \item a person may not notice a small difference between
the checked marking and the template one;
    \item the manual verification work is monotonous.
\end{itemize}

Existing projects are based on hardware solutions, for
example, on the use of special sensors [4].
\par
Such solutions implement marking recognition, but
have a number of important disadvantages:

\begin{itemize}
    \item Unstable recognition quality, depending on the conditions under which the recognition is performed.
Since the production line moves quickly, the necessary conditions for high-quality recognition are
often not met;
    \item Necessity to purchase specialized software to configure sensors.
\end{itemize} 

Thus, such solutions create additional difficulties in
operation, which are manifested in the need, in addition
to selective manual control of product quality, to control
the recognition system itself.

\begin{center}
    \Roman{chapter}. PROPOSED APPROACH
\end{center}

The proposed approach is to use a pipeline structure of
separate neural network modules, each of which solves
its own subproblem of marking recognition.
The task of this pipeline is to detect the marking,
determine its type and recognize it.
Let us stop on the system architecture in more detail.

\end{document}